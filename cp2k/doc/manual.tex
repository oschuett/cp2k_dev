% CP2K user manual (Release 1.0)
%
% Use the following command sequence to generate manual.pdf from
% manual.tex and manual.bib:
%
% pdflatex manual.tex
% bibtex manual
% makeindex manual
% pdflatex manual.tex
% pdflatex manual.tex
%
% History: - Creation (13.08.2002,MK)
%
\documentclass[12pt,twoside]{article}
%
\makeatletter
%
% Load readable fonts (screen), hypertext facilities, and index support
%
\usepackage{mathptmx}
\usepackage{times}
\usepackage[hyperindex]{hyperref}
\usepackage{makeidx}
\makeindex
%
% Page layout (twoside DIN A4 portrait with standard headings)
%
\setlength{\paperheight}{297mm}
\setlength{\paperwidth}{210mm}
\setlength{\textheight}{230mm}
\setlength{\textwidth}{150mm}
\setlength{\evensidemargin}{-0.4mm}
\setlength{\oddsidemargin}{9.6mm}
\setlength{\topmargin}{-0.4mm}
\setlength{\overfullrule}{2pt}
\setlength{\parindent}{0pt}
\setlength{\parskip}{0pt}
\renewcommand{\baselinestretch}{1.0}
\pagestyle{headings}
%
% Release data
%
\newcommand{\releasenumber}{1.0}
\newcommand{\releasedate}{\today}
%
% Some useful macros for typesetting keywords
% and the corresponding index entries
%
\newcommand{\kf}{\tt} % keyword font
\newcommand{\kw}[1]{{\kf #1}} % keyword
\newcommand{\iso}[1]{\index{input sections, optional!\kw{#1}}}
\newcommand{\isr}[1]{\index{input sections, required!\kw{#1}}}
\newcommand{\karg}[1]{\index{keyword arguments!\kw{#1}}}
\newcommand{\kopt}[1]{\index{keyword options!\kw{#1}}}
\newcommand{\ko}[1]{\index{keywords, optional!\kw{#1}}}
\newcommand{\kr}[1]{\index{keywords, required!\kw{#1}}}
%
% Modify some predefined environments
%
\renewenvironment{description}
 {\list{}{\setlength{\labelwidth}{0pt}
          \setlength{\leftmargin}{5mm}
          \setlength{\itemindent}{-\leftmargin}
          \let\makelabel\descriptionlabel}}
 {\endlist}
\renewcommand{\descriptionlabel}[1]{\hspace\labelsep\kf #1}
%
\renewenvironment{theindex}
 {\columnseprule\z@
  \columnsep 35\p@
  \twocolumn[\section*{\indexname}]%
  \@mkboth{\MakeUppercase\indexname}%
          {\MakeUppercase\indexname}%
  \parskip\z@ \@plus .3\p@\relax
  \let\item\@idxitem}
 {\onecolumn}
%
% Shorthands
%
\newcommand{\bc}{\begin{center}}
\newcommand{\ec}{\end{center}}
\newcommand{\bd}{\begin{description}}
\newcommand{\ed}{\end{description}}
\newcommand{\bi}{\begin{itemize}}
\newcommand{\ei}{\end{itemize}}
\newcommand{\la}{$\langle$}
\newcommand{\ra}{$\rangle$}
\newcommand{\nn}{\nonumber}
\newcommand{\op}[1]{{\bf \hat {\rm #1}}}
\newcommand{\sfrac}[2]{\mbox{$\frac{#1}{#2}$}}
%
% with 12pt there is no difference between \huge and \Huge, thus let's
% redefine \Huge with magnification 6 and add \HUGE with magnification 8
% with respect to 10pt
%
\newcommand\@xxxpt{29.87}
\renewcommand\Huge{\@setfontsize\Huge\@xxxpt{36}}
\newcommand\@xxxxiiipt{43.00}
\newcommand\HUGE{\@setfontsize\HUGE\@xxxxiiipt{52}}
%
\makeatother
%
\begin{document}
\thispagestyle{empty}
\bc
{\Huge User Manual}\\[50mm]
{\HUGE\bf CP2K}\\[20mm]
{\large A program package to perform\\[2mm]
        Molecular Dynamics Simulations}\\[25mm]
\href{http://developer.berlios.de/project/?group_id=129}
     {\Large The CP2K developers group}\\[50mm]
{\large CP2K program release \releasenumber\\[5mm]
        \releasedate}
\vfill
\href{http://www.ethz.ch}{ETH Zurich} --
\href{http://www.cscs.ch}{Centro Svizzero di Calcolo Scientifico (CSCS)}, Switzerland\\
\href{http://www.unizh.ch}{University of Zurich},
\href{http://pciwww.unizh.ch}{Physical Chemistry Institute}, Switzerland
\ec
%
\newpage
%
\thispagestyle{empty}
%
\subsection*{Disclaimer}
%
Please note that this manual is not complete. Basically it refers to the
CP2K program release \releasenumber, but the CP2K program package is
continuously improved and extended. Therefore the ultimate reference is
always the CP2K source code.\\[2mm]
Please cite this manual as:\\[2mm]
The CP2K developers group,
{\em CP2K User Manual (Release \releasenumber)}, Zurich, 2002
\vfill
%
\subsection*{Copyright}
%
CP2K: A program package to perform molecular dynamics simulations\\[2mm]
Copyright \copyright\ 2002
\href{http://developer.berlios.de/project/?group_id=129}
     {The CP2K developers group}\\[2mm]
This program is free software; you can redistribute it and/or modify
it under the terms of the
\href{http://www.gnu.org/copyleft/gpl.html}{GNU General Public License}%
\index{GNU}\index{General Public License}\index{GPL}
as published by the Free Software Foundation; either version 2 of the
License, or (at your option) any later version.
This program is distributed in the hope that it will be useful,
but WITHOUT ANY WARRANTY; without even the implied warranty of
MERCHANTABILITY or FITNESS FOR A PARTICULAR PURPOSE. See the
GNU General Public License for more details.
You should have received a copy of the GNU General Public License
along with this program; if not, write to the Free Software
Foundation, Inc., 675 Mass Ave, Cambridge, MA 02139, USA.
\newpage
%
\tableofcontents
\newpage\null\newpage
%
\section{Introduction}
%
The CP2K project was started in 2000 at the Max-Planck institute for
solid state research in Stuttgart. Now it is continued at the
ETH Zurich (CSCS) and at the University Zurich.

The current members of the CP2K developers group are
\bi
 \item Thomas Chassaing (University Zurich)
 \item Harald Forbert (University Bochum)
 \item J\"urg Hutter (University Zurich)
 \item Matthias Krack (ETH Zurich/CSCS)
 \item Fawzi Mohamed (ETH Zurich/CSCS)
 \item Christopher J. Mundy (LLNL)
 \item Ari P. Seitsonen (University Zurich)
 \item Gloria Tabacchi (Universit\`a degli studi dell'Insubria, Como)
 \item Joost VandeVondele (University Zurich)
\ei
\newpage
%
\section{Installation}
%
You can download the current version of the CP2K code from\\[2mm]
\href{http://developer.berlios.de/project/?group_id=129}
     {\tt http://developer.berlios.de/project/?group\_id=129}\\[2mm]
using CVS or FTP which also allow to update your current CP2K version.
Alternatively, you can directly download the full
\href{http://cvs.berlios.de/cgi-bin/viewcvs.cgi/cp2k/cp2k.tar.gz?tarball=1}
     {CP2K tarball} which you have to uncompress with\\[2mm]
{\tt \$ gunzip cp2k.tar.gz}\\[2mm]
Then extract the archive file with\\[2mm]
{\tt \$ tar -xvf cp2k.tar}\\[2mm]
In order to generate an executable change to the directory
with the makefile\\[2mm]
{\tt \$ cd cp2k/makefiles}\\[2mm]
and run GNU make which is on LINUX systems simply make\\[2mm]
{\tt \$ make}\\[2mm]
The default make will create a serial optimized (sopt) executable which
is equivalent to\\[2mm]
{\tt make sopt}\\[2mm]
Other choices are\\[2mm]
{\tt \$ make sdbg} (serial executable for debugging)\\[2mm]
{\tt \$ make pdbg} (parallel executable for debugging)\\[2mm]
{\tt \$ make popt} (optimized parallel executable)\\[2mm]
You can remove all the stuff generated by {\tt make} with\\[2mm]
{\tt \$ make distclean}\\[2mm]
If you want to remove only one version completely use e.g.\\[2mm]
{\tt \$ make sopt/realclean}\\[2mm]
or\\[2mm]
{\tt \$ make sopt/clean}\\[2mm]
to get rid of the object files only.
After a successful compilation you may find the corresponding executable
in the directory\\[2mm]
{\tt cp2k/exe/\la{\rm architecture name}\ra}\\[2mm]
There are some test inputs in the directory\\[2mm]
{\tt cp2k/tests}\\[2mm]
and the next section describes how to run the example inputs.
\newpage
%
\section{Running CP2K}
%
The CP2K program is started with the command\\[2mm]
{\tt \$ cp2k.sopt \la{\rm input file}\ra\ >\la{\rm output file}\ra}\\[2mm]
The start command for the parallel CP2K version depends on the parallel
execution environment of the underlying architecture, e.g.
with LINUX/MPICH you may start 4 processes with\\[2mm]
{\tt \$ mpirun -np 4 cp2k.popt \la{\rm input file}\ra\ >\la{\rm output file}\ra}\\[2mm]
whereas on IBMs (AIX) you have to type something like\\[2mm]
{\tt \$ poe cp2k.popt \la{\rm input file}\ra\ -procs 4 >\la{\rm output file}\ra}
%
\subsection{Input files}
%
The methods implemented in CP2K may require different additional input files.
%
\subsubsection{Fist}
%
\subsubsection{Quickstep}
%
\bi
 \item Potential file (default name: {\tt POTENTIAL})
  \index{files!potential}
 \item Basis set file (default name: {\tt BASIS\_SET})
  \index{files!basis set}
\ei
The Gaussian basis set format and all the atomic potential formats are
explained in the corresponding default database files.
\newpage
%
\section{Input description}
%
\subsection{General rules}
\index{input!general rules}
%
\bi
 \item {\bf Warning:} Do not expect the input to be logic.
       The programmers logic may be different from yours.
 \item {\bf Warning:} This input description may not refer to the
       actual version of the program you are using. Therefore the
       ultimate and authoritative input guide is the source code.
 \item The input is free format and is not case sensitive except where
       especially stated.
 \item Empty lines and white spaces at the beginning of a line are ignored.
 \item All characters in a line following the comment character are ignored.
       The default comment character is \#.
       \index{input!special characters}
 \item The CP2K input file is divided into input sections which are started
       and terminated with the section keywords listed below.
 \item Each section keyword starts with the section character.
       The default section character is \&.
 \item The order of the keywords inside an input section is arbitrary
       except where especially stated.
 \item For some keywords there are one or even more alias names which are
       given below as a comma-separated list.
 \item Lists enclosed in \{ \} imply that you have to choose
       {\bf exactly one} of the items.
 \item Lists enclosed in [ ] imply that you can choose
       {\bf any number} of items (optional keywords).
 \item There are several possibilities to define a floating point number
       \la real\ra, e.g. {\tt 0.05}, {\tt 5.0E-2}, {\tt 5.e-2},
       {\tt 1/20}, or {\tt 50/1000}. Also the specification of an
       integer number \la integer\ra\ is allowed where a floating point
       number \la real\ra\ is requested, but not vice versa.
 \item Strings \la string\ra\ with special characters like blanks have
       to be delimited by \kw{" "}.
\ei
%
\subsection{Section \&CP2K \ldots \&END}
\isr{\&CP2K}
%
The global CP2K section.
%
\subsubsection{Required keywords}
%
\bd
 \item[PROGRAM] \kw{\{FIST,QUICKSTEP\}}
  \kr{PROGRAM}\karg{QUICKSTEP}\karg{FIST}\\
  Defines the methods which is used for the calculation.
\ed
%
\subsubsection{Optional keywords}
%
\bd
 \item[FFTLIB] \kw{\{FFTESSL,FFTSG,FFTSGI,FFTW\}}
  \ko{FFTLIB}\karg{FFTESSL}\karg{FFTSG}\karg{FFTSGI}\karg{FFTW}\\
  Defines the library which is used for the Fast Fourier Transformations
  (FFT). The availability of the libraries depends on the architecture
  and/or installation, but at least the \kw{FFTSG} library is available
  which is included in the CP2K distribution.\\
  default: \kw{FFTSG}
 \item[IOLEVEL] \kw{\{\la{\rm integer}\ra\}}
  \ko{IOLEVEL}\\
  Global print level. See also input section \kw{\&PRINT}.\\
  default: 0
 \item[PP\_LIBRARY\_PATH] \kw{\{\la{\rm path name}\ra\}}
  \ko{PP\_LIBRARY\_PATH}\\
  Path to the directory with the pseudo potential database files.\\
  Default: \la the current working directory\ra
 \item[PROJECT] \kw{\{\la{\rm string}\ra\}}
  \ko{PROJECT}\\
  Name of the actual project.\\
  default: \kw{project}
\ed
%
\subsection{Section \&IO \ldots \&END}
\hyperdef{section}{IO}{}\iso{\&IO}
%
In this section the names of input and/or output files can be
modified.
\bd
 \item[BASIS\_SET\_FILE] \kw{\{\la{\rm file name}\ra\}}
  \ko{BASIS\_SET\_FILE}\\
  Name of the Gaussian basis set database file.\\
  default: \kw{BASIS\_SET}
 \item[POTENTIAL\_FILE] \kw{\{\la{\rm file name}\ra\}}
  \ko{POTENTIAL\_FILE}\\
  Name of the potential database file.\\
  default: \kw{POTENTIAL}
 \item[RESTART\_FILE] \kw{\{\la{\rm file name}\ra\}}
  \ko{RESTART\_FILE}\\
  Name of the restart file.\\
  default: \kw{RESTART}
\ed
%
\subsection{Section \&CELL \ldots \&END}
\isr{\&CELL}
%
This section is always needed to define the simulation cell.
%
\subsubsection{Required keywords}
%
\bd
 \item[ABC] \kw{\{\la{\rm real}\ra~\la{\rm real}\ra~\la{\rm real}\ra\}}
  \kr{ABC}\\
  Lengths of the vectors {\bf a}, {\bf b}, and {\bf c} which define the
  orthorhombic simulation cell. The unit of length is defined by the
  keyword \kw{UNIT}.
\ed
%
\subsubsection{Optional keywords}
%
\bd
 \item[UNIT] \kw{\{ANGSTROM,BOHR,SCALED\_ANGSTROM,SCALED\_BOHR\}}
  \ko{UNIT}\karg{ANGSTROM}\karg{BOHR}\karg{SCALED\_ANGSTROM}
           \karg{SCALED\_BOHR}\\
  Defines the unit of length for the simulation cell and it also
  applies to the definition of the atomic coordinates in the input
  section \kw{\&COORD}. Moreover, all lengths and distances in the
  output are printed using this unit.\\
  default: \kw{BOHR}
\ed
%
\subsection{Section \&COORD \ldots \&END}
\isr{\&COORD}
%
Each non-empty input line in this section defines an atom of the
considered system. The first entry in each line has to correspond
to an atomic kind name defined by a \kw{\&KIND} section which can
be a \la string\ra\ or an \la integer\ra\ number. The kind name has
to follow a set of three \la real\ra\ numbers defining the $x$, $y$,
and $z$ coordinates of the atom.
%
\subsection{Section \&KIND \ldots \&END}
\isr{\&KIND}
%
This section has to be defined for each atomic kind in a {\sc Quickstep}
run. The name of the kind has to be defined right after the {\kf \&KIND}
section keyword on the same input line. The kind name is referenced by the
{\kf \&COORD} section. Alternatively, the atomic number of the kind can be
defined as an integer number, e.g.\\[2mm]
{\kf \&KIND 6}\\[2mm]
for carbon which is equivalent to\\[2mm]
{\kf \&KIND C}\\[2mm]
In general, any \la string\ra\ can be defined for an atomic kind\\[2mm]
{\kf \&KIND \la{\rm string}\ra}\\[2mm]
which allows to define different atomic kinds for the same element
e.g. carbon with different orbital basis sets\\[2mm]
{\kf \&KIND C-DZVP}\\[2mm]
{\kf \&KIND C-TZVP}\\[2mm]
One or more {\kf \&KIND} sections are required for a {\sc Quickstep} run.
%
\subsubsection{Required keywords}
%
\bd
 \item[ORBITAL\_BASIS\_SET,BASIS\_SET,BAS] \kw{\{\la{\rm string}\ra\}}
  \kr{ORBITAL\_BASIS\_SET}\kr{BASIS\_SET}\kr{BAS}\\
  Name of the Gaussian orbital basis set which has to be read from the
  Gaussian basis set database file (see
  \hyperref{}{section}{IO}{section \kw{\&IO}}).
 \item[POTENTIAL,POT] \kw{\{\la{\rm string}\ra\}}
  \kr{POTENTIAL}\\
  Name of the atomic potential which has to be read from the potential
  database file (see \hyperref{}{section}{IO}{section \kw{\&IO}}).
\ed
%
\subsubsection{Optional keywords}
%
\bd
 \item[ELEMENT\_SYMBOL,ELEMENT] {\kf\{\la{\rm string}\ra\}}
  \ko{ELEMENT\_SYMBOL}\\
  Defines the element to which the atomic kind belongs.
 \item[ATOMIC\_MASS,MASS] {\kf\{\la{\rm real}\ra\}}
  \ko{ATOMIC\_MASS}\\
  Defines an atomic mass different from the default atomic mass, e.g.
  for the definition of isotopes.
 \item[PAO\_MIN\_BAS] {\kf\{{\rm list of \la integer\ra}\}}
  \ko{PAO\_MIN\_BAS}\\
  Definition of the projected atomic orbital (PAO) basis. Indices of
  the PAOs with respect to the full basis set.
\ed
%
\subsection{Section \&DFT \ldots \&END}
\iso{\&DFT}
%
In this section the configuration of a density functional calculation (DFT)
can be modified.
%
\bd
 \item[EXCHANGE-CORRELATION-FUNCTIONAL,XC-FUNCTIONAL,FUNCTIONAL]~\\
  \kw{\{BLYP,BP86,LDA,PADE,PBE,NONE\}}
  \ko{EXCHANGE-CORRELATION-\\FUNCTIONAL}\ko{XC-FUNCTIONAL}\ko{XC-FUN}
  \karg{PADE}\karg{LDA}\karg{PBE}\\
  Name of the requested exchange-correlation functional for a density
  functional calculation.\\
  default: \kw{PADE}
 \item[EXCHANGE-FUNCTIONAL,X-FUNCTIONAL] \kw{\{B88,SLATER,NONE\}}\\
  Name of the requested exchange functional.\\
  default: \kw{NONE}
 \item[CORRELATION-FUNCTIONAL,C-FUNCTIONAL] \kw{\{LYP,P86,NONE\}}\\
  Name of the requested correlation functional.\\
  default: \kw{PADE}
 \item[KINETIC-ENERGY-FUNCTIONAL,KE-FUNCTIONAL] \kw{\{NONE\}}\\
  Name of the requested kinetic energy functional.\\
  default: \kw{NONE}
 \item[DENSITY\_CUTOFF] \kw{\{\la{\rm real}\ra\}}\\
  Cutoff for the calculation of the density.\\
  default: \kw{1.0E-10}
 \item[GRADIENT\_CUTOFF] \kw{\{\la{\rm real}\ra\}}\\
  Cutoff for the calculation of the gradients.\\
  default: \kw{1.0E-8}
 \item[GRID] \kw{\{CUTOFF,MESH,PLANE\_WAVES\}}\\
  Definition of the integration grid.
  \bd
   \item[CUTOFF] \kw{\{\la{\rm real}\ra\}}\\
    Definition of the plane waves cutoff.
   \item[MESH] \kw{\{\la{\rm integer}\ra~\la{\rm integer}\ra~\la{\rm integer}\ra\}}\\
    Explicit definition of the grid size.
   \item[PLANE\_WAVES,PW]~\\
    default
  \ed
 \item[FORCES]~\\
  The calculation of the forces is requested.\\
  default: no force calculation
 \item[CHARGE] \kw{\{\la{\rm integer}\ra\}}\\
  The total charge of the system.\\
  default: \kw{0}
\ed
%
\subsection{Section \&QS \ldots \&END}
\iso{\&QS}
%
Program parameters {\sc Quickstep}
%
\bd
 \item[CUTOFF] \kw{\{\la{\rm real}\ra\}}
  \ko{CUTOFF}\\
  Plane waves cutoff of the largest grid in Rydberg.\\
  default: \kw{320}
 \item[EPS\_DEFAULT] \kw{\{\la{\rm real}\ra\}}
  \ko{EPS\_DEFAULT}\\
  Defines a default threshold value. All threshold values of {\sc Quickstep}
  are set to this value.
 \item[EPS\_CORE\_CHARGE] \kw{\{\la{\rm real}\ra\}}
  \ko{EPS\_CORE\_CHARGE}\\
  Threshold value for the interaction range of the atomic core charge
  distributions.\\
  default: \kw{1.0E-12}
 \item[EPS\_GVG\_RSPACE,EPS\_GVG] \kw{\{\la{\rm real}\ra\}}
  \ko{EPS\_GVG\_RSPACE}\ko{EPS\_GVG}\\
  Threshold value for the integration of the Hartree potential on the
  real space grid.\\
  default: \kw{1.0E-6}
 \item[EPS\_PGF\_ORB] \kw{\{\la{\rm real}\ra\}}
  \ko{EPS\_PGF\_ORB}\\
  Threshold value for interaction range of the primitive Gaussian-type
  orbital functions.\\
  default: \kw{1.0E-6}
 \item[EPS\_PPL] \kw{\{\la{\rm real}\ra\}}
  \ko{EPS\_PPL}\\
  Threshold value for the interaction range of the local part of the
  GTH pseudo potential.\\
  default: \kw{1.0E-12}
 \item[EPS\_PPNL] \kw{\{\la{\rm real}\ra\}}
  \ko{EPS\_PPNL}\\
  Threshold value for the interaction range of the non-local part of the
  GTH pseudo potential.\\
  default: \kw{1.0E-12}
 \item[EPS\_RHO] \kw{\{\la{\rm real}\ra\}}
  \ko{EPS\_RHO}\\
  Threshold values for \kw{EPS\_RHO\_GSPACE} and
  \kw{EPS\_RHO\_GSPACE}.\\
  default: \kw{1.0E-8}
 \item[EPS\_RHO\_GSPACE] \kw{\{\la{\rm real}\ra\}}
  \ko{EPS\_RHO\_GSPACE}\\
  Threshold value for the calculation of the electronic charge density in
  Fourier space.\\
  default: \kw{1.0E-8}
 \item[EPS\_RHO\_RSPACE] \kw{\{\la{\rm real}\ra\}}
  \ko{EPS\_RHO\_RSPACE}\\
  Threshold value for the calculation of the electronic charge density in
  real space.\\
  default: \kw{1.0E-8}
 \item[PROGRESSION\_FACTOR,PROFAC] \kw{\{\la{\rm real}\ra\}}
  \ko{PROGRESSION\_FACTOR}\ko{PROFAC}\\
  Progression factor for the generation of the multi-grid levels.\\
  default: \kw{2.0}
 \item[RELATIVE\_CUTOFF,REL\_CUTOFF] \kw{\{\la{\rm real}\ra\}}
  \ko{RELATIVE\_CUTOFF}\ko{REL\_CUTOFF}\\
  Relative plane waves cutoff for each multi-grid level. Values less
  than \kw{20.0} give inaccurate results and values greater than
  \kw{30.0} are used for reference calculations (save).\\
  default: \kw{25.0}
 \item[METHOD] \kw{\{GPW\}}
  \ko{METHOD}\karg{GAPW}\karg{GPW}\\
  Method used by {\sc Quickstep}. \kw{GAPW} is not available yet.\\
  default: \kw{GPW}
 \item[MULTI\_GRID] \kw{\{{\rm list of \la integer\ra}\}}
  \ko{MULTI\_GRID}\\
  The plane waves cutoffs for each multi grid level in Rydberg.
  The number of grid levels is defined by the keyword \kw{NGRID\_LEVEL}.
 \item[NGRID\_LEVEL,NGRID] \kw{\{\la{\rm integer}\ra\}}
  \ko{NGRID\_LEVEL}\ko{NGRID}\\
  Number of the multi-grid levels.\\
  default: \kw{4}
 \item[PAO]
  \ko{PAO}~\\
  Use the projected atomic orbital (PAO) method.\\
  default: no PAO
\ed
%
\subsection{Section \&SCF \ldots \&END}
\iso{\&SCF}
This section defines the parameters for the configuration of the self
consistent field (SCF) procedure which is used for the wavefunction
optimization.
%
\bd
 \item[ARPACK\_ON]
  \ko{ARPACK\_ON}\index{ARPACK}~\\
  The ARPACK eigensolver is used in a parallel run which requires a
  proper installation of the ARPACK library.\\
  default: no ARPACK usage
 \item[CHOLESKY\_ON,CHOLESKY\_OFF]
  \ko{CHOLESKY\_ON}\ko{CHOLESKY\_OFF}~\\
  Decides whether the Cholesky decomposition is used in the eigensolver
  or not.\\
  default: \kw{CHOLESKY\_ON}
 \item[DENSITY\_GUESS,SCF\_GUESS,GUESS] \kw{\{ATOMIC,CORE\}}
  \ko{DENSITY\_GUESS}\ko{SCF\_GUESS}\ko{GUESS}\karg{ATOMIC}\karg{CORE}~\\
  Defines the type of guess which is employed to generate the first density
  matrix.\\
  default: \kw{ATOMIC}
 \item[OT]
  \ko{OT}~\\
  An orbital transformation approach instead of a diagonalization is used
  for the wavefunction optimization during the SCF iteration procedure.
 \item[DENSITY\_MIXING,MIXING] \kw{\{\la{\rm real}\ra\}}
  \ko{DENSITY\_MIXING}\ko{MIXING}\\
  Factor for the mixing of the old and new density matrix during the
  wavefunction optimization.\\
  default: \kw{0.4} (i.e. 40\% of the new and 60\% of the old density are used)
 \item[EPS\_DIIS] \kw{\{\la{\rm real}\ra\}}
  \ko{EPS\_DIIS}~\\
  The DIIS procedure is switched on, if the maximum DIIS error vector
  element is below this threshold value.\\
  default: \kw{0.1}
 \item[EPS\_EIGVAL] \kw{\{\la{\rm real}\ra\}}
  \ko{EPS\_EIGVAL}\\
  Threshold value for eigenvector quenching when S$^{-1/2}$ is used as the
  orthogonalization matrix in the eigensolver.\\
  default: \kw{1.0E-5}
 \item[EPS\_SCF] \kw{\{\la{\rm real}\ra\}}
  \ko{EPS\_SCF}\\
  SCF convergence criterion, i.e. the maximum difference between the
  corresponding density matrix elements of two consecutive SCF iteration
  steps.\\
  default: \kw{1.0E-5}
 \item[LEVEL\_SHIFT] \kw{\{\la{\rm real}\ra\}}
  \ko{LEVEL\_SHIFT}\\
  Shift value for the unoccupied (virtual) molecular orbitals (MOs) in
  atomic units.\\
  default: \kw{0.0}
 \item[MAX\_DIIS] \kw{\{\la{\rm integer}\ra\}}
  \ko{MAX\_DIIS}\\
  Maximum size of the SCF DIIS buffer.\\
  default: \kw{0}
 \item[MAX\_SCF] \kw{\{\la{\rm integer}\ra\}}
  \ko{MAX\_SCF}\\
  Maximum number of SCF iteration steps.\\
  default: \kw{30}
 \item[NREBUILD] \kw{\{\la{\rm integer}\ra\}}
  \ko{NREBUILD}\\
  Number of SCF steps between two full calculations of the electronic
  charge density.
  default: \kw{1}
 \item[SMEAR] \kw{\{\la{\rm real}\ra\}}
  \ko{SMEAR}\\
  Window size in atomic units with respect to the eigenvalue of the highest
  occupied molecular orbital (HOMO) for the smearing of the occupation
  numbers.\\
  default: \kw{0.0}
 \item[WORK\_SYEVX] \kw{\{\la{\rm real}\ra\}}
  \ko{WORK\_SYEVX}\\
  Defines the amount of additional work space for the PDSYEVX
  routine\index{PDSYEVX} from the SCALAPACK library.
  A value between \kw{0.0} and \kw{1.0} is accepted\index{SCALAPACK}.
  (only for parallel runs using SCALAPACK and an eigensolver with
  diagonalization.\\
  default: \kw{0.0}
\ed
%
\subsection{Section \&PRINT \ldots \&END}
\iso{\&PRINT}
%
This sections allows for detailed output control when running {\sc Quickstep}.
There are 5 predefined print levels: 0, 1, 2, 3, and 4 which correspond to
the keywords \kw{NO}, \kw{LOW}, \kw{MEDIUM}, \kw{HIGH}, and \kw{DEBUG} or
\kw{FULL}. The print level has to be defined right after the \kw{\&PRINT}
section keyword on the same input line, e.g.\\[2mm]
\kw{\&PRINT LOW}\\[2mm]
which is equivalent to\\[2mm]
\kw{\&PRINT 1}\\[2mm]
The following keywords may be used based on the selected print level
to requested an additional output or to suppress an output selectively
by using the prefix \kw{NO\_} for the keyword, e.g. at print level
\kw{LOW} the atomic coordinates are listed in the output which may be
inconvenient for large sytems, thus simply request \kw{NO\_COORDINATES}
in the \kw{\&PRINT} section. The default print level is \kw{LOW}.
\bd
 \item[ANGLES]
  \ko{ANGLES}~\\
  Print the angles between all atom triples in the simulation cell.\\
  {\bf Warning:} That is much output for large systems.
 \item[ATOMIC\_COORDINATES,COORDINATES,COORD]
  \ko{ATOMIC\_COORDINATES}\ko{COORDINATES}\ko{COORD}~\\
  Print all atomic coordinates together with the some atomic kind information.
 \item[BASIC\_DATA\_TYPES]~\\
  Print informations about the basic data types like {\tt REAL},
  {\kf INTEGER}, or {\tt LOGICAL}.
 \item[BASIS\_SETS,BASIS\_SET,BASIS]
  \ko{BASIS\_SETS}\ko{BASIS}~\\
  Print the Gaussian basis set information, i.e. all Gaussian function
  exponents and the corresponding contraction coefficients as read
  from the Gaussian basis set da\-ta\-ba\-se file. Furthermore, the
  normalized contraction coefficients are printed.
 \item[BLACS\_INFO]
  \ko{BLACS\_INFO}~\\
  Print the process grid information of
  BLACS (Basic linear algebra subprograms)
 \item[CARTESIAN\_MATRICES]
  \ko{CARTESIAN\_MATRICES}~\\
  Print all operator matrices in the Cartesian instead of the spherical
  representation.
 \item[CELL\_PARAMETERS,CELL]
  \ko{CELL\_PARAMETERS}\ko{CELL}~\\
  Print the simulation cell data like the cell vectors, cell volume etc.
 \item[CORE\_HAMILTONIAN\_MATRIX,H\_MATRIX]
  \ko{H\_MATRIX]}~\\
  Print the core Hamiltonian matrix.
 \item[CORE\_CHARGE\_RADII,CORE\_RADII]
  \ko{CORE\_CHARGE\_RADII}\ko{CORE\_RADII}~\\
  Print the radius of the core charge distribution for each atomic kind.
 \item[DENSITY\_MATRIX,P\_MATRIX]
  \ko{DENSITY\_MATRIX}\ko{P\_MATRIX}~\\
  Print the density matrix.
 \item[DERIVATIVES]
  \ko{DERIVATIVES}~\\
  Print the first derivatives of the operator matrices.
 \item[DFT\_CONTROL\_PARAMETERS]
  \ko{DFT\_CONTROL\_PARAMETERS}~\\
  Print the DFT control parameters as defined in the \kw{\&DFT} section.
 \item[DIIS\_INFORMATION]
  \ko{DIIS\_INFORMATION}~\\
  Print information about the SCF DIIS procedure.
 \item[DISTRIBUTION]
  \ko{DISTRIBUTION}~\\
  Print the distribution and the sparsity of the overlap matrix (only
  parallel version).
 \item[EACH\_SCF\_STEP]
  \ko{EACH\_SCF\_STEP}~\\
  Print the requested energies, densities, or matrices for each SCF iteration
  step.
 \item[E\_DENSITY\_CUBE]
  \ko{E\_DENSITY\_CUBE}~\\
  Print the electronic charge density as a cube file.
 \item[FORCES]
  \ko{FORCES}~\\
  Print the atomic force contributions for all atoms.
 \item[HOMO]
  \ko{HOMO}~\\
  Print the highest occupied molecular orbital (HOMO) as a cube file.
 \item[INTERATOMIC\_DISTANCES,DISTANCES]
  \ko{INTERATOMIC\_DISTANCES}\ko{DISTANCES}~\\
  Print a matrix with the interatomic distances.\\
  {\bf Warning:} That is much output for large systems.
 \item[KIND\_RADII]
  \ko{\kf KIND\_RADII}~\\
  Print the maximum interaction radius of each atomic kind.
 \item[KINETIC\_ENERGY\_MATRIX,T\_MATRIX]
  \ko{KINETIC\_ENERGY\_MATRIX}\ko{T\_MATRIX}~\\
  Print the kinetic energy integral matrix.
 \item[KOHN\_SHAM\_MATRIX]
  \ko{KOHN\_SHAM\_MATRIX}~\\
  Print the Kohn-Sham matrix.
 \item[LUMO]
  \ko{LUMO}~\\
  Print the lowest unoccupied molecular orbital (LUMO) as a cube file.
 \item[MEMORY]
  \ko{MEMORY}~\\
  Print informations about the memory usage of the CP2K program.
 \item[MO\_EIGENVALUES]
  \ko{MO\_EIGENVALUES}~\\
  Print the eigenvalues of the molecular orbitals (MOs).
 \item[MO\_EIGENVECTORS,MOS]
  \ko{MO\_EIGENVECTORS}\ko{MOS}~\\
  Print the eigenvectors, eigenvalues, and the occupation numbers of the
  molecular orbitals (MOs).
 \item[MO\_OCCUPATION\_NUMBERS]
  \ko{MO\_OCCUPATION\_NUMBERS}~\\
  Print the occupation numbers and the eigenvalues of the molecular
  orbitals (MOs).
 \item[NEIGHBOR\_LISTS]
  \ko{NEIGHBOR\_LISTS}~\\
  Print all neighbor lists.\\
  {\bf Warning:} That is much output for large systems.
 \item[ORTHO\_MATRIX]
  \ko{ORTHO\_MATRIX}~\\
  Print the orthogonalisation matrix used to transform the Kohn-Sham matrix.
 \item[OVERLAP\_MATRIX]
  \ko{OVERLAP\_MATRIX}~\\
  Print the overlap matrix.
 \item[PGF\_RADII]~\\
  Print the interaction radii of all primitive Gaussian-type functions.
 \item[PHYSICAL\_CONSTANTS,PHYSCON]~\\
  Print the values of all physical constants used in the program.
 \item[POTENTIALS]
  \ko{POTENTIALS}~\\
  Print a detailed atomic potential information for each atomic kind.
 \item[PPL\_RADII]
  \ko{PPL\_RADII}~\\
  Print the interaction radii of the local part of the
  Goedecker-Teter-Hutter (GTH) pseudo potential \cite{GTH:1996,GTH:1998}.
 \item[PPNL\_RADII]
  \ko{PPNL\_RADII}~\\
  Print the interaction radii of the non-local projector functions of the
  Goedecker-Teter-Hutter (GTH) pseudo potential \cite{GTH:1996,GTH:1998}.
 \item[PROGRAM\_BANNER]
  \ko{PROGRAM\_BANNER}~\\
  Print a program banner.
 \item[PROGRAM\_RUN\_INFORMATION]
  \ko{PROGRAM\_RUN\_INFO}~\\
  Print informations about the current program run.
 \item[PW\_GRID\_INFORMATION]
  \ko{PW\_GRID\_INFORMATION}~\\
  Print detailed informations about the used plane waves grid.
 \item[RADII]
  \ko{RADII}~\\
  Print all interaction radii for each atomic kinds.
 \item[SCF]
  \ko{SCF}~\\
  Print the SCF iteration.
 \item[SCF\_ENERGIES]
  \ko{SCF\_ENERGIES}~\\
  Print all contributions to the total SCF energy.
 \item[SET\_RADII]
  \ko{SET\_RADII}~\\
  Print the interaction radii of all Gaussian orbital sets.
 \item[SPHERICAL\_HARMONICS]
  \ko{SPHERICAL\_HARMONICS}~\\
  Print the transformation matrices between Cartesian and spherical function.
 \item[TIMING\_INFORMATION]
  \ko{TIMING\_INFORMATION}~\\
  Print timing information depending on the \kw{IOLEVEL} defined in the
  \kw{\&CP2K} section.
 \item[TITLE]
  \ko{TITLE}~\\
  Print the title.
 \item[TOTAL\_DENSITIES]
  \ko{TOTAL\_DENSITIES}~\\
  Print
 \item[TOTAL\_NUMBERS]
  \ko{TOTAL\_NUMBERS}~\\
  Print the total number of atoms, shell sets, basis functions, projectors etc.
 \item[V\_HARTREE\_CUBE]
  \ko{ V\_HARTREE\_CUBE}~\\
  Print the Hartree potential as a cube file.
 \item[W\_MATRIX]
  \ko{W\_MATRIX}~\\
  Print the energy weighted density matrix used for the force calculation.
\ed
\newpage
%
\section{Input examples}
\index{input!examples}
%
\subsection{Argon atom}
%
\small
\begin{verbatim}
&CP2K
 PROGRAM      Quickstep
 IOLEVEL      10
 FFTLIB       FFTSG
&END

&DFT
 FUNCTIONAL   PADE
&END

&QS
 CUTOFF       300
 EPS_DEFAULT  1.0E-12
 EPS_RHO      1.0E-8
 EPS_GVG      1.0E-6
 REL_CUTOFF   30
&END

&SCF
 GUESS        ATOMIC
 EPS_DIIS     0.1
 MAX_DIIS     4
 EPS_SCF      1.0E-6
 MAX_SCF      30
 MIXING       0.4
&END

&PRINT medium
NO_BLACS_INFO
&END

&KIND Ar
 BASIS_SET    DZVP-GTH-PADE
 POTENTIAL    GTH
&END

&CELL
 UNIT         ANGSTROM
 ABC          12.0  12.0  12.0
&END

&COORD
 18     0.000000  0.000000  0.000000
&END
\end{verbatim}
\normalsize
\newpage
%
\subsection{Water molecule}
%
\small
\begin{verbatim}
&CP2K
 PROGRAM      Quickstep
 IOLEVEL      10
 FFTLIB       FFTSG
&END

&DFT
 FUNCTIONAL   Pade
 FORCES
&END

&QS
 CUTOFF       200
&END

&SCF
 GUESS        ATOMIC
 MIXING       0.4
 EPS_SCF      1.0E-5
&END

&PRINT medium
&END

&KIND H
 BASIS_SET    DZV-GTH-PADE
 POTENTIAL    GTH
&END

&KIND O
 BASIS_SET    DZVP-GTH-PADE
 POTENTIAL    GTH
&END

&CELL
 UNIT         ANGSTROM
 ABC          10.0  10.0  10.0
&END

&COORD
 H   0.000000   -0.757136    0.520545
 O   0.000000    0.000000   -0.065587
 H   0.000000    0.757136    0.520545
&END
\end{verbatim}
\normalsize
\newpage
%
\section{Methods}
%
\subsection{GPW method}
%
The electronic energy functional for a molecular or crystalline system in
the framework of the Gaussian Plane Waves (GPW) method is defined as
\cite{GPW}
\begin{eqnarray}
 \label{GPW-GTH-PP}
 E^{\rm el}[n] &=& E^{\rm T}[n] + E^{\rm V}[n] + E^{\rm H}[n] +
                   E^{\rm XC}[n] \nn \\
               &=& \sum_{\mu\nu} P_{\mu\nu}
                    \langle
                     \varphi_\mu({\bf r})
                     \mid -\sfrac{1}{2}\nabla^2 \mid
                     \varphi_\nu({\bf r})
                    \rangle + \nn \\
               & & \sum_{\mu\nu} P_{\mu\nu}
                    \langle
                     \varphi_\mu({\bf r})
                     \mid V_{\rm loc}^{\rm PP}(r) \mid
                     \varphi_\nu({\bf r})
                    \rangle -\\
               & & \sum_{\mu\nu} P_{\mu\nu}
                    \langle
                     \varphi_\mu({\bf r})
                     \mid V_{\rm nl}^{\rm PP}({\bf r},{\bf r}^\prime) \mid
                     \varphi_\nu({\bf r}^\prime)
                    \rangle + \nn \\
               & & 4\pi\,\Omega \sum_{|{\bf G}| < G_{\rm C}}
                    \frac{\tilde n^*({\bf G})\,
                          \tilde n({\bf G})}{{\bf G}^2} + \nn \\
               & & \int\tilde n({\bf r})\,\varepsilon_{\rm XC}[\tilde n]\,
                   d{\bf r}
\end{eqnarray}
where $E^{\rm T}[n]$ is the kinetic energy, $E^{\rm V}[n]$ is the electronic
interaction with the ionic cores, $E^{\rm H}[n]$ is the electronic Hartree
(Coulomb) energy and $E^{\rm XC}[n]$ is the exchange-correlation energy.

The electronic density
\[
 n({\bf r}) = \sum_{\mu\nu} P_{\mu\nu}\varphi_\mu({\bf r})\varphi_\nu({\bf r})
\]
is expanded in a set of contracted Gaussian functions
\[
 \varphi_\mu({\bf r}) = \sum_i C_{i\mu} g_i({\bf r})
\]
$P_{\mu\nu}$ is a density matrix element, $g_i({\bf r})$ is a primitive
Gaussian function and $C_{i\mu}$ is the corresponding contraction coefficient.

An auxiliary basis set of plane waves is used as an intermediate basis set
to describe the electronic charge density
\[
 \tilde n({\bf r}) = \frac{1}{\Omega}\sum_{|{\bf G}| < G_{\rm C}}
                     n({\bf G}) {\rm e}^{i {\bf Gr}}
\]
which is used for the calculation of the density dependent contributions
$E^{\rm H}[n]$ and $E^{\rm XC}[n]$.
$\Omega$ is the volume of the periodic cell. The plane wave expansion is
truncated by the specification of a cutoff value for the kinetic energy
\[
 E_{\rm C} = \frac{1}{2} G_{\rm C}^2
\]
of the plane waves.
Since the $G = 0$ term of the Hartree energy is treated with the Ewald method,
the nuclear charges are represented by a Gaussian charge distribution and not
by point charges.

The GPW method works like pure plane waves methods with atomic pseudo
potentials, since an expansion of Gaussian functions with large exponents
is numerically not efficient or even not feasible.
The current implemention of the GPW method uses only the pseudo potentials
of Goedecker, Teter and Hutter (GTH) \cite{GTH:1996,GTH:1998} which
consist of a local part $V_{\rm loc}^{\rm PP}(r)$ and a non-local part
$V_{\rm nl}^{\rm PP}({\bf r},{\bf r}^\prime)$ as shown in
Eq. \ref{GPW-GTH-PP}.
\newpage
%
\bibliographystyle{plain}
\bibliography{manual}
%
\printindex
%
\end{document}
